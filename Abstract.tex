During the last years the main cause of mortality in Mexico and around the world is due to cardiovascular diseases (CVD's), because of the high levels of obesity and hypertension problems among the population. The auscultation is an easy and economic method to diagnose cardiac anomalies, mostly those diseases associated with the malfunction of some of the heart valves. 

The phonocardiogram signal (PCG), recorded during auscultation, contains information in low frequencies that are not perceptible by the human auditory system. This problem originated the appearance of digital stethoscopes, devices that allow to visualize the electric waveform of cardiac sounds. Whit this advance it is posible to analyze the PCG and its time-frequency features as a complement to diagnose heart diseases. 

However, the traditional audio coders have good resolution and support to high frequency components of the signal, giving an excess of resources in phonocardiogram coding and making hard to transmit this signals over low data rate networks. For this reason, in this thesis we develop an audio codec adapted to cardiac sounds, representing efficiently its low frequency components, giving a low bit rate, signal compression and the amount of information necessary to reconstruct it with accuracy. 

In this framework we have exploited the Matching Pursuit Method with Gabor dictionaries and Linear Predictive Coding to the modelling, representation and parametric coding for phonocardiogram signals. The codifier is evaluated by subjective tests as well as objective measures verifying the quality of the reconstructed signal.





