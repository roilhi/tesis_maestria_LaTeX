%TCIDATA{Version=5.00.0.2552}
%TCIDATA{LaTeXparent=0,0,Tesis.tex}

Durante los últimos años la principal causa de mortalidad en México y a nivel mundial está dada por las enfermedades cardiovasculares (CVD's) debido a los altos índices de obesidad y problemas de hipertensión entre la población. La auscultación es un método económico y sencillo de diagnóstico para la atención de anomalías cardíacas, sobre todo aquellas relacionadas al mal funcionamiento en alguna de las válvulas asociadas al corazón. 

Los límites del oído humano a las bajas frecuencias que presenta una señal de fonocardiograma (PCG) obtenida mediante la auscultación han originado el surgimiento de estetoscopios electrónicos, con los cuales es posible visualizar la forma de onda eléctrica del audio cardiaco. Con este avance es posible el análisis tiempo-frecuencia del PCG complementando y reforzando el diagnóstico de patologías cardiacas.

No obstante en la actualidad los codificadores tradicionales de audio tienen soporte y buena resolución para las altas frecuencias, proporcionando un exceso de recursos en la codificación de un fonocardiograma y complicando su transmisión sobre redes de bajas tasas de datos. Por tal motivo, se propone en este trabajo de tesis implementar un codificador-decodificador adaptado al audio cardíaco, a su rango definido de frecuencias y a baja tasa de datos, proporcionando compresión y la información necesaria para que sea reconstruido con calidad.

En este trabajo de tesis se han explotado las técnicas de Matching Pursuit con diccionarios de Gabor y de codificación predictiva lineal (LPC) para el modelado, representación y codificación paramétrica de una señal de fonocardiograma. El codificador es evaluado mediante pruebas subjetivas y objetivas verificando la calidad de la señal reconstruida. 


